% 这个文件是用来定义其他 Latex 文档基本的、通用的样式的
% 中文文档
\documentclass{ctexart}

% 使用超链接宏包
\usepackage{hyperref}
% 设置纸张的宏包
\usepackage{geometry}
% 用于绘图的宏包
\usepackage{tikz}
% 用于书写数学公式的宏包
\usepackage{amsmath}
% 用户设置字体的宏包
\usepackage{fontspec}

% 定义纸张大小为 A4,内容所占区域为 95%
\geometry{a4paper, scale=0.95}
% 定义超链接样式
\hypersetup{
    CJKbookmarks=true, % 支持中文书签
    colorlinks=true,
    citecolor=blue,
    linkcolor=blue,
    urlcolor=blue,
    breaklinks=true % 允许链接中的换行
}

% 设置字体
% 设置英文字体
\setmainfont{JetBrains Mono}
% 设置中文字体
\setCJKmainfont{130-SS Zhui Guang Shou Xie Ti}

%作者
\author{aszswaz}


\title{数学-名词定义}
\date{2022-03-22}

\begin{document}
\maketitle

% 微积分学名词定义
\section{微积分学}

\subsection{\href{https://zh.wikipedia.org/wiki/\%E5\%AF\%BC\%E6\%95\%B0}{导数}}
\textbf{导数}(英语:derivative)是微积分学中的一个概念。函数在某一点的导数是指这个函数在这一点附近的变化率。导数的本质是通过极限的概念对函数进行局部的线性逼近。

导数是函数的局部性质。导数与斜率的概念有关,我们可以求两个端点连成的线段的斜率:

\begin{center}
	\begin{tikzpicture}
	\end{tikzpicture}
\end{center}

\end{document}
