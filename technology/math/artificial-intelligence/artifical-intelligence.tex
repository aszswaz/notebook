% 定义文档类型:中文文章
\documentclass{ctexart}

% 导入宏包
\usepackage{tikz}
\usetikzlibrary{positioning}

% 文档标题
\title{人工智能-基础篇}
\author{aszswaz}
\date{2022-03-17}

\begin{document}
\maketitle

\section{Rosenblatt 感知器}
Rosenblatt 感知器是第一个从算法上完整描述的神经网络。假设 $w$ 为斜率、$t$ 为标准答案、$e$ 为误差、$a$ 为实数,感知器的流程图如下:\\
% 绘制函数流程图
\begin{tikzpicture}[
    % 有圆角的长方形
    squarednode/.style={rectangle, draw=black, minimum size=5mm, rounded corners}
]
    %Nodes
    \node[squarednode]      (maintopic)                              {输入$x$};
    \node[squarednode]      (gety)     [below=of maintopic]          {$y = w \cdot x$};
    \node[squarednode]      (error)    [below=of gety]               {$t - y = e$};
    \node[squarednode]      (neww)     [below=of error]              {$n = w + a \cdot e \cdot x$};
    \node[squarednode]      (assgin)   [below=of neww]               {$w = n$};

    %Lines
    \draw[->] (maintopic)       -- (gety);
    \draw[->] (gety)            -- (error);
    \draw[->] (error)           -- (neww);
    \draw[->] (neww)            -- (assgin);
    \draw[->] (assgin.west)     .. controls +(left: 20mm) and +(left: 20mm) .. (gety.west);
\end{tikzpicture}

\end{document}
