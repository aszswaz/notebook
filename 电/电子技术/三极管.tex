\documentclass{ctexart}

% circuitikz 用于绘制电路图,支持的电路图符号有 american 和 european
\usepackage [american] {circuitikz}

\begin{document}
NPN 三极管的接线电路图:
    \begin{center}
        \begin{circuitikz}
            \def\showcoord(#1){coordinate(#1) node[circle, red, draw, inner sep=1pt,
                pin={[red, overlay, inner sep=0.5pt, font=\tiny, pin distance=0.1cm,
                pin edge={red, overlay}]45:#1}](){}}
        
            % 5V 和 3.3V 电源
            \draw (0, 0) to[isource, v = $5V$] (0, 6) -- (2, 6);
            \draw (2.77, 0) to[isource, v = $3.3V$] (2.77, 3);
            % 3.3V 电源的电阻
            \draw (2.77, 3) to[R = $10KR$] (2.77, 5.5);
            \draw (0, 0) -- (2.77, 0);
            
            % 三极管
            \draw (2.77, 6) node[npn, rotate = 90](NPN){};
            \path (NPN.C) \showcoord(C) (NPN.B) \showcoord(B) (NPN.E) \showcoord(E);
            
            % 发光二极管
            \draw (3.5, 6) -- (6, 6);
            \draw (6, 6) to[full led, i = $20MA$] (6, 4) -- (6, 3);
            \draw (6, 3) to[R = $150R$] (6, 0) -- (2.77, 0);
        \end{circuitikz}
    \end{center}
\end{document}